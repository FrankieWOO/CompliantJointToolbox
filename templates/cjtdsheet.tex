%\documentclass[letterpaper, 9 pt, conference]{article}  % Comment this line out if you need a4paper
\documentclass[a4paper,10pt]{cjtdsheet}      % Use this line for a4 paper

\usepackage{lipsum}

\begin{document}
\renewcommand{\arraystretch}{1.2}
\actuatorname{\valJointName}

\begin{multicols}{2}
\begin{tabularx}{0.95\columnwidth}[c]{p{3cm}lXr}
   \rowcolor{cjtblue}
   \textcolor{white}{\textbf{Mechanical}} 
   & \textcolor{white}{\textbf{Symbol}} 
   & \textcolor{white}{\textbf{Unit}} 
   & \textcolor{white}{\textbf{Value}} 
  \tabularnewline
   %
        Transmission Ratio                  & $\symGearratio$      &  -                    & \valGearratio           \tabularnewline \rowcolor{lightgray}
        Mass                                & $\symMass $          & kg                    & \valMass                \tabularnewline 
        Rotor Inertia                       & $\symInertiarotor$   & $\text{kg}\text{m}^2$ & \valInertiarotor        \tabularnewline \rowcolor{lightgray}
        Gear Inertia                        & $\symInertiagear$    & $\text{kg}\text{m}^2$ & \valInertiagear         \tabularnewline 
        Spring Inertia                      & $\symInertiaspring$  & $\text{kg}\text{m}^2$ & \valInertiaspring       \tabularnewline \rowcolor{lightgray}
        Diameter                            & $\symDiameter$       & mm                    & \valDiameter            \tabularnewline 
        Length                              & $\symActlength$      & mm                    & \valActlength           \tabularnewline \rowcolor{lightgray}
        Mechanical Time \newline Constant   & $\symTmech$          & s                     & \valTmech               \tabularnewline 
        Viscous Friction                    & $\symViscousdamping$ & Nm s/rad              & \valViscousdamping      \tabularnewline \rowcolor{lightgray}
        Coulomb Friction                    & $\symCoulombdamping$ & Nm                    & \valCoulombdamping      \tabularnewline 
%
%
%
    \rowcolor{cjtblue}
    \textcolor{white}{\textbf{Electrical}}   
        & \textcolor{white}{\textbf{Symbol}} 
        & \textcolor{white}{\textbf{Unit}} 
        & \textcolor{white}{\textbf{Value}}
    \tabularnewline
        Winding \newline Resistance        & $\symArmatureresistance$  & $\Omega$          & \valArmatureresistance   \tabularnewline    \rowcolor{lightgray}
        Winding \newline Inductance        & $\symArmatureinductance$ &  mH               & \valArmatureinductance   \tabularnewline
        Electrical \newline Time Constant  & $\symTel$                &  s                & \valTel                  \tabularnewline    \rowcolor{lightgray}
        Torque Constant                    & $\symTorqueconstant$     &  Nm/A             & \valTorqueconstant       \tabularnewline    
        Generator \newline Constant        & $\symGeneratorconstant$  &  Vs/rad           & \valGeneratorconstant    \tabularnewline    \rowcolor{lightgray}
%
    \rowcolor{cjtblue}
    \textcolor{white}{\textbf{Thermal}} 
        & \textcolor{white}{\textbf{Symbol}} 
        & \textcolor{white}{\textbf{Unit}} 
        & \textcolor{white}{\textbf{Value}} 
    \tabularnewline
    Max. Temperature                    & $\symTmpWindMax$     & $^\circ$C   & \valTmpWindMax         \tabularnewline     \rowcolor{lightgray}
    Winding \newline Time Constant      & $\symTthw$           &  s          & \valTthw               \tabularnewline     
\end{tabularx}

%%%
%%%
%%%
\begin{tabularx}{0.95\columnwidth}[c]{p{3cm}lXr}
%
%
%
    Housing \newline Time Constant        	   & $\symTthm$               &  s          & \valTthm                \tabularnewline     \rowcolor{lightgray}
    Therm. Resistance \newline Winding-Housing & $\symResthermWH$         &  K/W        & \valResthermWH          \tabularnewline     
    Therm. Resistance \newline Housing-Air     & $\symResthermHA$         &  K/W        & \valResthermHA          \tabularnewline     \rowcolor{lightgray}
    \rowcolor{cjtblue}
    \textcolor{white}{\textbf{Rated Operation}} 
        & \textcolor{white}{\textbf{Symbol}} 
        & \textcolor{white}{\textbf{Unit}}
        & \textcolor{white}{\textbf{Value}}
    \tabularnewline
    Rated Voltage                   & $\symRatedvoltage$         & V                & \valRatedvoltage        \tabularnewline     
    Rated Current                   & $\symRatedcurrent$         & A                & \valRatedcurrent        \tabularnewline    \rowcolor{lightgray}
    Rated Torque                    & $\symRatedtorque $         & Nm               & \valRatedtorque         \tabularnewline    
    Rated Speed                     & $\symRatedspeed  $         & rad/s            & \valRatedspeed          \tabularnewline    \rowcolor{lightgray}
    Rated El. Power                 & $\symRatedpowere $         & W                & \valRatedpowere         \tabularnewline    
    Rated Mech. Power               & $\symRatedpowerm $         & W                & \valRatedpowerm         \tabularnewline    \rowcolor{lightgray}
    No Load Current                 & $\symNoloadcurrent$        & A                & \valNoloadcurrent       \tabularnewline    
    No Load Torque                  & $\symNoloadtorque $        & Nm               & \valNoloadtorque        \tabularnewline    \rowcolor{lightgray}
    No Load Speed                   & $\symNoloadspeed  $        & rad/s            & \valNoloadspeed         \tabularnewline    
    Stall Torque                    & $\symStalltorque  $        & Nm               & \valStalltorque         \tabularnewline    \rowcolor{lightgray}
    Starting Current                & $\symStartingcurrent$      & A                & \valStartingcurrent     \tabularnewline    
    Torque Speed \newline Gradient  & $\symSpeedtorquegradient$  & rad/Nms          & \valSpeedtorquegradient \tabularnewline    \rowcolor{lightgray}
%
%
%
    \rowcolor{cjtblue}
    \textcolor{white}{\textbf{Peak Operation}}   
        & \textcolor{white}{\textbf{Symbol}} 
        & \textcolor{white}{\textbf{Unit}} 
        & \textcolor{white}{\textbf{Value}} 
    \tabularnewline
    Peak Current                    & $\symMaxcurrent$ & A                 & \valMaxcurrent     \tabularnewline     
    Peak Torque                     & $\symMaxtorque$  & Nm                & \valMaxtorque      \tabularnewline     \rowcolor{lightgray}
    Peak Speed                      & $\symMaxspeed $  & rad/s             & \valMaxspeed       \tabularnewline     
    Peak El. Power                  & $\symMaxpowere$  & W                 & \valMaxpowere      \tabularnewline     \rowcolor{lightgray}
    Peak Mech. Power                & $\symMaxpowerm$  & W                 & \valMaxpowerm      \tabularnewline 
%
%   Max. Efficiency                 & $\eta_{max}$         & \%                   & \valMaxefficiency      \tabularnewline     \rowcolor{lightgray} 
    \end{tabularx}

\end{multicols}

\begin{figure}[b!]
	\centering
	\caption{Torque-speed diagram of the actuator.}
    \includegraphics[width=\textwidth]{\valTorqueSpeedFName}
	\label{fig:TorqueSpeedCurve}
\end{figure}

\newpage

\section*{\textcolor{cjtred}{Parameter Definitions}}
The parameters listed in the table on the previous page are briefly explained in the following.

\begin{multicols}{2}

%
%------------------------------------------------------
%
\subsection*{\textcolor{cjtred}{Mechanical Properties}}
The mechanical properties define the outer shape of the actuator and the transmission of power generated by the motor to the load.

\emph{Transmission Ratio:} The actuator typically comprises a transmission mechanism such as a gear box, that amplifies the torque output trading off the deliverable speed. The value is the ratio of the transmission input speed to the obtained output speed. Conversely, it describes the ratio of the output torque over the input torque.

\emph{Mass:} The actuator comprises of three main parts: the motor, the transmission mechanism and the torque sensor. The value is the summed mass of all three components.

\emph{Rotor Inertia:} The rotary inertia of the motor shaft.

\emph{Transmission Inertia:} The rotary inertia of the transmission mechanism.

\emph{Spring Inertia:} The rotary inertia of the deflective element used for torque sensing.

\emph{Diameter:} The maximum diameter of the actuator considering all components: the motor, transmission mechanism and torque sensor.

\emph{Length:} The overall length of the actuator including the motor, transmission mechanism and torque sensor.

\emph{Mechanical Time Constant:} Describes the acceleration of the actuator when a constant voltage is applied, under the condition that the required starting current can be provided. The value is the time for the actuator output to arrive at 63 \% of the steady state velocity for the applied voltage ignoring friction and additional load. The mechanical time constant is computed to:
\begin{equation}
\symTmech = \frac{
	\left(
		  \symInertiarotor 
		+ \symInertiagear 
		+ \symInertiaspring
	\right)
	\symArmatureresistance
	}
	{
	 \symTorqueconstant^2
	}.
\end{equation}

\emph{Viscous Friction:} Coefficient for the speed dependent friction torque. The value is an approximation of the effect. Due to individual assembly and manufacturing tolerances of each actuator, the actual friction torque at a given speed will differ. Moreover the friction torque varies with temperature, lubrication changes, load, wear and tear.

\emph{Coulomb Friction:} Constant friction torque. The constant friction torque is subject to the same variations and tolerances as the viscous friction coefficient.

%
%------------------------------------------------------
%
\subsection*{\textcolor{cjtred}{Electrical Properties}}
The electrical properties describe the windings, which form the core element of the actuator. 

\emph{Winding Resistance:} Manufacturing tolerances (wire diameter) can cause variations in the order of 8~\%. The value is given for the windings with leads at a temperature of $25^\circ$C. The resistance variation with respect to a temperature increase by $\Delta \nu$ is:
\begin{equation}
	\symArmatureresistance(\nu) = \symArmatureresistance(25^\circ\text{ C})\:\left(1+\alpha_{CU} 	\: \Delta \nu\right)	\:.
\end{equation}
The temperature coefficient of copper $\alpha_{CU}$ is 0.0039~$K^{-1}$. As a consequence the Joule losses increase with higher winding temperatures.
 
\emph{Winding Inductance:} The inductance of the motor windings. The effective inductance can vary with the frequency of the current signal. The apparent inductance can be smaller when the motor is operated using pulse-width modulation.

\emph{Electrical Time Constant:} Describes the time for the motor current to reach the 63~\% of the steady state value, when a constant voltage is applied to the windings. The value is computed from the winding inductance and resistance to:
\begin{equation}
\symTel = \symArmatureinductance / \symArmatureresistance\:.
\end{equation}

\emph{Torque Constant:} Describes the potential of the winding current to generate an electrical torque:
\begin{equation}
	\tau = \symTorqueconstant \: i\:.
\end{equation}
The electrical torque is the amplified by the transmission system. The torque constant is determined by the rotor magnet. The rotor magnetisation has a tolerance in the order of 8~\%. Moreover magnetisation weakens with increasing temperature. The effect is reversible as long as the winding current does not exceed the admissible peak current.

\emph{Generator Constant:} Relates the voltage induced in the windings to the rotor speed:
\begin{equation}
	v_{ind} = \symGeneratorconstant \: \dot{\theta}.
\end{equation}
The generator constant is the inverse of the torque constant $\symTorqueconstant$.

%
%------------------------------------------------------
%
\subsection*{\textcolor{cjtred}{Thermal Properties}}
The thermal properties represent a simplified and practical set of parameters to model the thermodynamics of the actuator. They are the foundation for the definition of the current and power ratings as well as the intermittend operating times beyond the limit of the rated operation.

\emph{Maximum Temperature:} Defines the maximum winding temperature before heat induced irreversible damage occurs.

\emph{Thermal Winding Time Constant:} Time for the motor windings to arrive at 63~\% of the steady state temperature when a constant current is applied to the windings.

\emph{Thermal Housing Time Constant:} Time for the motor housing to arrive at 63~\% of the steady state temperature when a constant current is applied to the windings.

\emph{Thermal Resistance Winding-Housing:} Thermal resistance to a heat flow between the motor windings and housing. Defines the winding temperature increase for a given Joule power loss.

\emph{Thermal Resistance Housing-Air:} Thermal resistance to a heat flow between the motor housing and the surrounding environment. Defines the housing temperature increase for a given Joule power loss. The value is typically obtained with standard experiment. In that experiment the motor is flanged to a plastic plate with the motor body being free and sorrounded by air.

%
%------------------------------------------------------
%
\subsection*{\textcolor{cjtred}{Rated Operation}}
The rated parameters together with the No-Load parameters frame the conditions at which the actuator can be operated continuously without risking any damage.

\emph{Rated Voltage:} Serves as a reference for the definition of further actuator parameters. It is recommended but not mandatory to operate the actuator at the rated voltage. The rated voltage determines the speed axis offset of the torque-speed curve in Fig.~\ref{fig:TorqueSpeedCurve}. 

\emph{Rated Current:} Describes the maximum continuous current that does not result in thermal damage to the motor. The value is given with respect to pure Joule losses. At higher speeds, internal losses such eddy currents and magnetization losses in the stator contribute to the heat generation and reduce the actually continuously permissible current.

\emph{Rated Torque:} The electrical torque generated by rated current $\symRatedcurrent$.

\emph{Rated Speed:} The speed obtained from the torque-speed curve at rated torque. 

\emph{Rated Electrical Power:} Computed as a product of rated voltage $\symRatedvoltage$ and rated current $\symRatedcurrent$.

\emph{Rated Mechanical Power:} Computed as the product of the rated torque $\symRatedtorque$ and rated speed $\symRatedspeed$.

\emph{No-Load Current:} Steady state winding current when the rated voltage is applied to the load free actuator. The current produces the torque required to overcome the actuator friction.

\emph{No-Load Torque:} Torque that is generated by the No-Load current $\symNoloadcurrent$. It is equivalent to the friction torque.

\emph{No-Load Speed:} Rotor speed that emerges when the rated voltage is applied to the load free actuator. The theoretical No-Load speed $\dot{\theta}_0$ for a frictionless actuator would be computed from the rated voltage $\symRatedvoltage$ and the generator constant $\symGeneratorconstant$:
\begin{equation}
	\dot{\theta}_0 = \symRatedvoltage / \symGeneratorconstant\:.
\end{equation}

\emph{Stall Torque:} The load torque at which the motor stops moving, if the rated voltage $\symRatedvoltage$ is applied to the windings.

\emph{Starting Current:} The winding current corresponding to the stall torque $\symStalltorque$. When applying the rated voltage $\symRatedvoltage$ to the actuator at rest, the initial current corresponds to starting current, if the power supply can provide it.

\emph{Torque-Speed Gradient:} Slope of the torque-speed curve.

%
%------------------------------------------------------
%
\subsection*{\textcolor{cjtred}{Peak Operation}}
The peak operation parameters define the maximum intermittend operating conditions. Exceeding these values even for short time yields irriversible damage to the actuator.

\emph{Peak Current:} The maximum permissible current that does not lead to a demagnetization of the permanent magnet. If the winding current exceeds this value irreversible weakening of the motor occurs.

\emph{Peak Torque:} Torque corresponding to the peak current $\symMaxcurrent$.

\emph{Peak Speed:} Maximum permissible intermittend speed for the actuator. This value is determined mostly by the mechanical supports, bearings and transmission mechanism. Exceeding this speed leads to irreversible damage of the mechanics.

\emph{Peak Electrical Power:} Computed as a product of rated voltage $\symRatedvoltage$ and peak current $\symMaxcurrent$.

\emph{Peak Mechanical Power:} Computed as the product of the peak torque $\symMaxtorque$ and peak speed $\symMaxspeed$.

\end{multicols}

\end{document}